Compared to JavaScript, BWS overcomes JavaScript's limitations and enables any task within the browser that can be done with a supported scripting language and that doesn't violate the safety measures of the Java sandbox. BWS, using BSF, also provides access to the full Java class library within the user's preferred scripting language. And in comparision to ActiveScripting, BWS allows an as easy integration of scripts and documents, but in a browser and operating system neutral and more secure way.

BWS scripts can be edited ad hoc an immediatly tested in a web browser, without the need to compile them or do anything besides editing; if they are not embedded directly in the BWS/HTML document but referenced, it is not even necessary to reload the document in the browser.

The future of BWS to a large part depends on the future development of Java and web browsers. One of the most important task is the use of the Common DOM API instead of LiveConnect. The Common DOM API provides a direct, DOM conform, way of accessing the DOM of the document currently loaded in the web browser. The Common DOM API was originally planned to be available with Java 1.4. However, Java 1.4 only provided a extremely limited and practically unusable version of the Common DOM API. Sun, despite announcing to do so, did not change that in later releases of Java 1.4 and now plans to have a working version with Java 1.5 which will probably be available in the third quarter of 2004.
%quelle!

For ease of use, it is also planned to provide a server-side rewriting mechanism in form of PHP and JSP scripts that would do document rewriting on the server without any interaction from the user and without the need to embed the applet in the document by hand.