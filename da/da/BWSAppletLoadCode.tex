Loading the script code is done using the \texttt{private} method \texttt{getScript()}: This method expects a \texttt{JSNode} of the script and returns the script code as a \texttt{String}.

The script code can be either embedded between the \texttt{script} tages or referenced specifying the scripts URL as the \texttt{src} attribute of the \texttt{script} tag. The URL may either be specified absolute or relative to the location of the current document.

The first action of the \texttt{getScript()} method is to determine if the script is embedded or referenced: This is done by checking if the \texttt{script} element has got a \texttt{src} attribute that is not \texttt{null} or empty. If it has not got one, the text within the script tag is read using the \texttt{JSNode getInnerHTML()} method and returned. If a \texttt{src} attribute is available, the content of the \texttt{script} tag is ignored and the script tries to read the script code from the specified URL.

If the script is referenced, it has to be determined if the specified path is relative to the BWS/HTML document's path or if it is an absolute path: This is done by checking if the value of the \texttt{src} attribute contains a colon followed by a double forward slash (\texttt{://}). Values of \texttt{src} containing this sequence are treated as absolute pathes while method not containing it are treated as relative pathes.

\textsc{Sequenz muss noch im Applet ge�ndert werden! Im Moment nur Doppelpunkt. URL durch Pfad ersetzen! Evtl. �berpr�fen ob / am Anfang, dann direkt nach Webserver anfangen!}

If the path specified is detected as a relative path, the applets URL is read using the \texttt{Applet getCodeBase()} method. The result of this method is converted to a \texttt{String}. The URL of the document is then obtained concatenating the applet's URL with the relative specified as the \texttt{src} attribute.

Reading the script then works just like any other stream reading, i.e. a stream is created on the URL, read to a byte array and a \texttt{String} is created from this array. This \texttt{String} then contains the script code and is returned.
