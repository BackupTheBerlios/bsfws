ActiveScripting, also called ActiveX Scripting, is a technology developed by Microsoft that can be used for client side scripting in web browsers. It is a proprietary solution available only for the Microsoft platform, that is for the Windows operating systems and the Internet Explorer web browser.

ActiveScripting is based on COM interfaces that can be used to communicate with the various supported scripting engines. It is not limited to a specifig scripting language, instead scripting engines for various languages are available from Microsoft and other vendors, see table  \ref{tab:ActiveScriptingEngines} for an overview of some available languages.\footnote{Developing a scripting language for languages other than these is possible, information on this is available at \cite{MsCreateSE}.}

%ref MsCreateSE: http://msdn.microsoft.com/library/default.asp?url=/library/en-us/script56/html/scripting.asp

\begin{table}[hb]
	\centering
		\begin{tabular}{ll}
			\textbf{Language}& \textbf{Vendor} \\
			& \textbf{URL} \\
			VBScript & Microsoft \\
			& \texttt{http://www.microsoft.com/scripting/vbscript/} \\
			JScript & Microsoft \\
			& \texttt{http://www.microsoft.com/scripting/jscript/} \\
			Perl & ActiveState \\
			& \texttt{http://www.activestate.com/Products/ActivePerl/}\\
			Python & Python.org \\
			&\texttt{http://www.python.org/windows/win32com/ActiveXScripting.html} \\
		%	Haskell & \sffamily http://www.haskell.org/haskellscript \\
			Object Rexx & IBM \\
			& \texttt{http://www-306.ibm.com/software/awdtools/obj-rexx/}\\
			
		\end{tabular}
	\caption{Active Scripting Engines}
	\label{tab:ActiveScriptingEngines}
\end{table}


Used as a client side web scripting language, some of the advantages of ActiveScripting are:

\begin{description}
	
	\item [Language Independent] ActiveScript not depending on a specific programming language developers must use; instead, it provides the possibility to choose one of several available languages like JScript, VBScript or Perl.
	
	\item [Powerful] ActiveScripting languages are based on the windows COM and can therefore use all parts of a system that provide COM interfaces, including system ressources and most Windows applications.
	
	\item [Easy to Use] ActiveScripting languages can access the DOM of HTML documents works exactly as this works using standard JavaScript scripts\footnote{Microsoft's JScript itself is an ActiveScripting language.}, the \emph{learning barrier} to using non-JavaScript languages in HTML environments is very low.

\end{description}

In contrast to these advantages, ActiveScripting also has some severe disadvantages:

\begin{description}
  
  \item [Insecure] ActiveScripting within HTML documents can only be switched on or off\footnote{The decision if scripting shall be allowed can be made based on signatures and URLs.}, a differentiated approach to allow or prohibit specific actions of a script is not possible. It also is impossible to run scripts in a secure environment similar to the Java sandbox.
  
  \item [Platform Dependence] ActiveScripting is limited to the Microsoft Windows/Internet Explorer platform and thus the creation of applications based on this technology creates massive lock-in effects and raises switching costs to other platforms as solutions based on this platform would have to be rewritten at least partially.
  
\end{description}