JavaScript is the common name for the scripting language defined in ECMA standard 262. It was originally developed as JavaScript by Brendan Eich at Netscape. The purpose of JavaScript was to give web developers the posibility do enable the scripting of HTML documents and was therefore integrated with Netscape's Navigator since version 2.0. Microsoft based its own solutions for scriptable HTML documents on this language and called the resulting derivative JScript. Microsoft extended this solution to be language independent and \emph{called the result of this project ActiveScripting}, this approch will be detailed in section \ref{sec:activescripting}. JScript was included in the Internet Explorer starting with 3.0.

The development of the standard for the language the standardized as ECMAScript started in November 1996; the standard was adopted first in June 1997. It was also accepted as ISO/IEC standard 16262 in April 1998. As of February 2004, the standard is in its third edition and availabe from \cite{ECMA-262}.

%ECMA262: http://www.ecma-international.org/publications/standards/Ecma-262.htm
%	+ pdf document

JavaScript/ECMAScript to different extents is now available in almost all web browsers; additionally, stand-alone JavaScript interpreters exist that allow the use of JavaScript as a \emph{system} script language. The most prominent of these interpreters is the Rhino JavaScript interpreter maintained by and available from The Mozilla Organisation.

%mozilla.org, mozilla.org/rhino, availablity of JS in browsers?

For use as client side scripting language in web applications, JavaScript has some important disadvantages:

\begin{itemize}
	\item The usability of JavaScript is quite limited, it doesn't support any more sophisticated tasks like file system or database access. 
	\item Another problem with Java\-Script based solutions is that although Java\-Script is standardized, different browsers interpret the same JavaScript code differently and any non-trivial application based on JavaScript has to be customized to the browser platform it will be used on. A consequence of this is that JavaScript applications get overly complex and expensive when the resulting application shall be availble on multible browsers.
	
\end{itemize}

Advantages of JavaScript are:
\begin{itemize}
	\item JavaScript is very easy to learn, it is not a strongly typed language, the syntax is familiar for any C/Java/Perl/PHP programmer.
	\item JavaScript is by default integrated in almost any available browser and thus provides a wide base users that can use JavaScript applications 'out of the box'.
	\item JavaScript, in the standardized version, is quite secure, especially because it does not support any potentially hazardous actions like deleting or reading files or similar.
\end{itemize}