XPath, standarized by the W3C is a XML element query language: It allowes to specify certain criteria that are then used to select all elements of a XML document that match these criteria. The easiest way is specifying the path to an element, to address the \texttt{h1} element in figure \ref{fig:simpleDomTree}, the apropriate XPath expression would be \verb|/html/body/h1|, that is, the root element (\verb-/-) and all relevant children types separated by (forward) slashes.

For selecting the \texttt{h1} content, the procedure would be this:

\begin{enumerate}
	\item Select the \texttt{h1} element with XPath using \verb|/html/body/h1|.
	\item Get this element's content.
\end{enumerate}

However, this way of selecting elements is still based on the full path of the elements within the DOM tree and does not allow accessing elements only by type (without specifying a full path). For this purpose, XPath provides the posibility to select elements independent from their path. XPath expressions using this method start with a double slash (\verb|//|) instead of the single root slash. The expression for selecting all \texttt{h1} elements in a document therefore would be \verb|//h1|.

This would allow to use the following procedure for selecting the \texttt{h1} content:

\begin{enumerate}
	\item Select all \texttt{h1} elements with XPath using \verb|//h1|.
	\item Get all selected elements' content.
\end{enumerate}

The difference between the two examples using XPath above is that the second example would return any \texttt{h1} element within the document independant of its position (i.e. it would also return a \texttt{h1} lying under the \texttt{p} element in the given document whereas the first expression would only reply \texttt{h1} elements that are exacly in the third layer of the document and have \texttt{html/body} as parents.