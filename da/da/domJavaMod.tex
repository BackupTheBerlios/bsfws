The first step in modifying an existing\footnote{dom4j also provides the possibility to build DOM trees/XML documents form scratch. As this possibility is not used in BWS, it is not \emph{detailed} in this paper.}  DOM in Java is parsing the DOM underlying XML document and building the DOM tree in Java. dom4j uses the Simple API for XML (SAX)\footnote{SAX is available from \cite{SaxHP}} for this purpose.

%saxhp=www.saxproject.org

Reading the XML document is done by using a \texttt{org.dom4j.io.SAXReader} objects \texttt{read()} method on the source of the XML document; the source may be a \texttt{String}, an \texttt{URL}, a \texttt{InputStream}, a \texttt{Reader} or a \texttt{org.sax.InputSource}. This method returns a DOM tree as an \texttt{org.dom4j.Document}.

The root element of the parsed document can then be obtained using the \texttt{getRootElement()} method on the document returned from the \texttt{SAXReader}. Elements in dom4j are objects of the class \texttt{org.dom4j.Element}.

To allow access to child elements of the current element, each \texttt{Element} provides a standard Java \texttt{Iterator} that allows iteration over all child objects. Figure \ref{fig:AMinimalDom4jExample} shows a minimal dom4j example that provides two methods: \texttt{parseDocFromURL()} to read an XML document from an URL and \texttt{printRootChildren()} to print the element types of all direct child nodes of the root element of the document read with the \texttt{parseDocFromURL()} method.

\begin{figure}[htb]
  \begin{verbatim}
import java.util.Iterator;
import java.net.URL;

import org.dom4j.*;
import org.dom4j.io.SAXReader;

public class Dom4jExample {
  private Document dom4jDocument;
  
  /* SAXReader.read() throws a DocumentException when problems reading
   * or parsing an XML document occur, for example if the document is
   * not available at the specified place or if it is not well-formed.
   */
  public void parseDocFromURL(URL anURL) throws DocumentException {
    SAXReader xmlReader = new SAXReader();
    this.dom4jDocument = xmlReader.read(anURL);
  }
  
  public void printRootChildren() {
    Element documentRoot = this.dom4jDocument.getRootElement();
    
    Iterator elementIterator = documentRoot.elementIterator();
    
    // iterate over all children
    while(elementIterator.hasNext()) {
      Element currentElement = (Element)elementIterator.next();
      System.out.println(currentElement.getName());
    }
  }
}
  \end{verbatim}

Once the desired element is found, its attributes 

	\caption{A minimal dom4j example}
	\label{fig:AMinimalDom4jExample}
\end{figure}

As stated above, this iterative approach is not practicable for larger documents or documents with a previously unknown structure. In these cases, the XPath support of dom4j is helpful. It allows to select elements based on XPath querys and returns the results of this queries as a \texttt{java.util.Collection}. A (shortened) example of printing the number of \texttt{h1} elements in a document is given in figure \ref{fig:UsingXPathWithDom4j}.

\begin{figure}[htbp]
	
\begin{verbatim}
import org.dom4j.XPath;
import org.dom4j.DocumentHelper;

...

  // create XPath query
  XPath xpathSelector = DocumentHelper.createXPath("//h1");
  
  // select elements from document
  List results = xpathSelector.selectNodes(document);
  
  // print number of elements
  System.out.println(results.size());
  
... 
\end{verbatim}
	\caption{Using XPath with dom4j}
	\label{fig:UsingXPathWithDom4j}
\end{figure}

% anleitung s. dom4j cookbook


Once the desired element has been found, several methods are available for modifying the elements itself, its attributes and its content. See table \ref{tab:dom4jElementMethods} for an overview of the most important ones.

\begin{table}[htbp]
	\centering
		\begin{tabular}{lll}
			Return Value & Method Head & Description \\
			\texttt{void} & \texttt{addAttribute(String, String)} & sets the specified attribute to the specified value.\\
			\texttt{Attribute} & \texttt{attribute(int)} & returns the specified attribute.\\
			\ttfamily Attribute & \texttt{attribute(String)} & returns the specified attribute.\\
			\texttt{List} & \texttt{attributes()} & returns as a backed\footnote{I.e. any change in the attribute is 
			  done directly to the referenced Attribute, not to a clone.} \texttt{List}.\\
			\texttt{String} & \texttt{attributeValue(String)} & Same as \texttt{attribute(String)}, but returns the value instead of the attribute itself.\\
			\texttt{String} & \texttt{getText()} & returns the text value of the element.\\
			\texttt{void} & \texttt{setAttributes(List)} & Sets the attributes of the element.\\
			\texttt{void} & \texttt{setText(String)}\footnote{Inherited unchanged from \texttt{org.dom4j.Node}} & Sets the Text of the Attribute\\
			
			  
		\end{tabular}
	\caption{dom4j Element Methods}
	\label{tab:dom4jElementMethods}
\end{table}