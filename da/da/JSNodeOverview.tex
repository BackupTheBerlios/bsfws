

A \texttt{JSNode} object can be constructed using one of three available constructors, see table \ref{tab:JSNodeConstructors}.

\begin{table}[htbp]
	\centering
		\begin{tabular}{ll}
			\textbf{Method Head} & \textbf{Description} \\
			\texttt{JSNode(JSObject window,} & Creates a reference to the node with the \texttt{id nodeRef}\\
			\texttt{ String nodeRef)}&  in the \texttt{window}.\\
			
			\texttt{JSNode(JSObject window,}& Creates a reference to the node passed as \texttt{existingNode}.\\
			\texttt{ JSObject existingNode)}& \\
			
			\texttt{JSNode(JSObject window,}& Creates a reference to the node passed as \texttt{exitingNode}\\
			\texttt{ JSObject existingNode,}&  and sets its \texttt{id} to the specified \texttt{id}.\\
			\texttt{ String id)}& \\
		\end{tabular}
	\caption{JSNode Constructors}
	\label{tab:JSNodeConstructors}
\end{table}

Additionaly, a node can be obtained without specifing a window calling an existing \texttt{JSNode}'s \texttt{getJSNode()} method. This method expects a reference to a existing node of the type \texttt{JSObject} as its argument. It returns a new JSNode with the window of the called node.

%texttt{JSNode} also provides methods

Other methods provided by \texttt{JSNode} can be used to obtain the nodes window (\texttt{getWindow()} and document (\texttt{getDocument()}), its \texttt{id}(\texttt{getIdentifier()}), the full content in HTML of an element (\texttt{getInnerHTML()}) and the node object referenced by the \texttt{JSNode} (\texttt{getNode()}).

\texttt{JSNode} also provides three methods for creating new nodes of different types:

\begin{itemize}
	\item \texttt{createAttribute(String attributeType, String attributeValue)}, creates an attribute named \texttt{attributeType}, e.g. \texttt{style} or \texttt{name}, set to the value \texttt{attributeValue}.
	\item \texttt{createElement(String elementType)}, creates an element of the type \texttt{elementType}, e.g. \texttt{h1} or \texttt{div}.
	\item \texttt{createTextNode(String elementText)}, creates a text node with the specified content (\texttt{elementText}).
\end{itemize}

