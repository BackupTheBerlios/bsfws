The \texttt{executeScript()} method, available in \emph{two overloaded ways}, is the core of the BWS runtime environment. It is called from the BWS/HT\-ML document by the JavaScript/DOM event handlers and upon calling, runs the scripts specified with the method call.

It can be called either with one or with two parameters. If it is called with one parameter, this parameter must be a \texttt{String} specifying a script call string\footnote{Cf. section \ref{sec:scriptString}.}; the two parameter variant expects an DOM node, represented as a \texttt{JSObject}, as additional second parameter. This second parameter may be accessed within the script string's parameters section using \texttt{this} as a shortcut.\footnote{Cf. section \ref{sec:docRewritting}.}

%unterschied zw. ein u. zwei parameter variante

The first step in the execution of BWS script is the interpretation of the script call string passed with the script call. This is done using the \texttt{ScriptString} class. After the script \texttt{id} has been determined by \texttt{ScriptString}, it is used to load the appropriate scripting engine and look up the script code using the \texttt{private loadScriptingEngine()} and \texttt{getScript()} methods.

After the engine has been loaded and the script is available, the script parameters are evaluated. This is done first reading the parameters keys from the \texttt{ScriptString} with \texttt{getParameters()} and the passing the resulting array to another \texttt{private BWSApplet} method: \texttt{evaluateParameters()} which maps parameter keys to the according objects and returns a \texttt{Vector} containing the parameter objects.\footnote{Cf. section \ref{sec:mapParamObject}.}

Next, an argument name \texttt{Vector} is created that contains the names of the arguments, as the original names of these objects are not known, they are named \texttt{argumentX} where \texttt{X} is an index starting at zero.

The final steps before the script is eventually executed are the lookup of the return key using \texttt{ScriptString}'s \texttt{getRegKey()} method and the creation of an \texttt{Object} reference that will hold the return value of the script execution.

Script execution is done using the \texttt{apply()} method provided by the \texttt{BSFEngine}. This method expects six parameters, \emph{description123} the first three of which are left empty respective zero in BWS, the fourth is a \texttt{String} containing the complete script code, the fifth and sixth are the \texttt{Vector}'s containing the names of the parameters passed and the parameters for the script. The methods return value is stored in the \texttt{Object} created before script creation.

After script execution, the returned object as well as the return key is checked if it contains a value or \texttt{null}. If both contain values, the object is stored in the BSF registry under the specified key and returned; if one or both are \texttt{null} the object is only returned to the script caller.

